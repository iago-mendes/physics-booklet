\chapter{Classical Mechanics}

\section{Kinematics}
    \subsection{Basic concepts}
        \begin{itemize}
            \item Velocity
                \begin{equation*}
                    \vt{v}=\frac{d\v{r}}{dt}
                \end{equation*}
                $\vt{r}$: position
            \item Speed
                \begin{equation*}
                    v=\vm{v}=\left|\frac{d\v{r}}{dt}\right|
                \end{equation*}
            \item Acceleration
                \begin{equation*}
                    \vt{a}=\dt{\v{v}}
                \end{equation*}
            \item Other rates
                \begin{itemize}
                    \item Jerk
                        \begin{equation*}
                            \vt{j}=\dt{\v{a}}
                        \end{equation*}
                    \item Snap
                        \begin{equation*}
                            \vt{s}=\dt{\v{j}}
                        \end{equation*}
                    \item Crackle
                        \begin{equation*}
                            \vt{c}=\dt{\v{s}}
                        \end{equation*}
                    \item Pop
                        \begin{equation*}
                            \vt{p}=\dt{\v{c}}
                        \end{equation*}
                \end{itemize}
        \end{itemize}
    \subsection{Constant acceleration cases (theorems)}
        \begin{enumerate}
            \item $\t{v}=v_0+at$
                \begin{equation}
                    a = \dfrac{dv}{dt} \qtq \it a dt=\it \dfrac{dv}{dt} dt \\
                    \tq a \cdot (t-t_ 0)=\left[\t{v}\right]_{t_0}^t  \qtq \t{v} = v_0 + at
                \end{equation}
            \item $\t{x} = x_0 + v_0t + \frac{at^2}{2}$
                \begin{equation}
                    \t{v} = v_0 + at \qtq \it \t{v} dt = \it (v_0 + at)dt \\
                    \tq \t{x} = x_0 + v_0t + \frac{at^2}{2}
                \end{equation}
            \item $\t{x} = x_0 + vt - \frac{at^2}{2}$
                \begin{equation}
                    v = v_0 + at \qtq v_0=v-at \\
                    \tq \t{x} = x_0 + (v-at)t + \frac{at^2}{2} \qtq \t{x} = x_0 + vt - \frac{at^2}{2}
                \end{equation}
            \item $\t{x} = x_0 + \frac{(v_0+v)t}{2}$
                \begin{equation}
                    2x = \t{x}+\t{x} = \left( x_0 + v_0t + \frac{at^2}{2} \right) + \left(x_0 + vt - \frac{at^2}{2}\right) \\
                    \tq 2x = 2x_0 + v_0t + vt \qtq \t{x} = x_0 + \frac{(v_0+v)t}{2}
                \end{equation}
            \item $v^2 = v_0^2 + 2a(x-x_0)$
                \begin{equation}
                    \begin{cases} \t{v} = v_0 + at \\
                    \t{x} = x_0 + \frac{(v_0+v)t}{2} \end{cases}
                    \qtq \begin{cases}
                        v-v_0=at \\
                        v+v_0 = \frac{2(x-x_0)}{t}
                    \end{cases} \\
                    \tq v^2 = v_0^2 + 2a(x-x_0)
                \end{equation}
        \end{enumerate}
    \subsection{Uniform circular motion}
        \begin{itemize}
            \item Angular velocity
                \begin{equation}
                    \omega=\dt{\theta}
                \end{equation}
            \item Position
                \begin{equation}
                    \vt{r} = R\cos(\omega t)\hat{i}+R\sin(\omega t) \hat{j}
                \end{equation}
            \item Velocity
                \begin{equation}
                    \vt{v}=-\omega R\sin(\omega t)\hat{i}+\omega R\cos(\omega t)\hat{j}
                \end{equation}
            \item Speed
                \begin{equation}
                    v=\omega R
                \end{equation}
            \item Centripetal acceleration
                \begin{equation}
                    \v{a}=-\omega^2\v{r} \\
                    a=\omega^2R=\frac{v^2}{R}
                \end{equation}
        \end{itemize}
\section{Forces}
    \subsection{Newton's Laws}
        \begin{enumerate}
            \item Inertia
                \begin{center}
                    Every object moves in a straight line unless acted upon by a force.
                \end{center}
            \item $F=ma$
                \begin{equation}
                    \v{F}_{net}=\sum \v{F}=m\v{a}
                \end{equation}
            \item Action and reaction
                \begin{center}
                    For every action, there is an equal and opposite reaction
                \end{center}
        \end{enumerate}
    \subsection{Weight $|$ Near-Earth gravitional force $(W)$}
        \begin{itemize}
            \item Definition
                \begin{equation}
                    \v{W}=-mg\hat{k}
                \end{equation}
            \item Gravity
                \begin{equation}
                    g\approx9.81 \, \frac{m}{s^2} \quad \textrm{(downward)}
                \end{equation}
        \end{itemize}
    \subsection{Tension $(T)$}
        \begin{itemize}
            \item Definition
                \begin{center}
                    Pulling force transmitted axially by the means of a rope to keep it from changing its length.
                \end{center}
            \item Ideal rope
                \begin{itemize}
                    \item massless
                    \item doesn't stretch or break
                \end{itemize}
        \end{itemize}
    \subsection{Normal force $(N)$}
        \begin{itemize}
            \item Definition
                \begin{center}
                    Contact force orthogonal to a surface that keeps two solid objects from passing through each other.
                \end{center}
        \end{itemize}
    \subsection{Friction}
        \begin{itemize}
            \item Definition
                \begin{center}
                    Resistance to sliding at an interface.
                \end{center}
            \item Static friction
                \begin{equation}
                    \vm{F_s}\leq\mu_sN
                \end{equation}
                $\mu_s$: coefficient of static friction
            \item Kinectic friction
                \begin{equation}
                    \vm{F_k}=\mu_kN
                \end{equation}
                $\mu_k$: coefficient of kinectic friction
            \item General relation between constants
                \begin{equation}
                    \mu_s>\mu_k
                \end{equation}
        \end{itemize}
    \subsection{Drag $(D)$}
        \begin{itemize}
            \item Viscous force (linear drag)
                \begin{equation}
                    D\propto v
                \end{equation}
            \item Air resistance (quadratic drag)
                \begin{equation}
                    D=\frac{1}{2}C\rho Av^2
                \end{equation}
                \begin{flushleft}
                    $C$: drag coefficient (associated with shape) \linebreak
                    $\rho$: mass density of air \linebreak
                    $A$: cross-section surface area
                \end{flushleft}
            \item Terminal speed
                \begin{equation}
                    v_t=\sqrt{\frac{2mg}{C\rho A}}
                \end{equation}
        \end{itemize}
    \subsection{Spring force}
        \begin{itemize}
            \item Hooke's law
                \begin{equation}
                    F=-kx
                \end{equation}
        \end{itemize}

\section{Energy}
    \subsection{Basic concepts}
        \begin{itemize}
            \item Work $(W)$
                \begin{equation}
                    W = \int \v{F}\bullet d\v{r}
                \end{equation}
            \item Kinetic energy $(K)$
                \begin{equation}
                    K=\frac{1}{2}mv^2
                \end{equation}
            \item Potential energy $(U)$
                \begin{equation}
                    \Delta U = - \int \v{F_{cons}}\bullet d\v{r} = - \Delta W_{cons}
                \end{equation}
                \begin{itemize}
                    \item Spring
                        \begin{equation}
                            U=\frac{1}{2}kx^2
                        \end{equation}
                    \item Weight
                        \begin{equation}
                            U=mgz
                        \end{equation}
                \end{itemize}
            \item Work-Energy theorem
                \begin{equation}
                    W_{net}=\Delta K
                \end{equation}
                \begin{itemize}
                    \item Derivation
                        \begin{equation}
                            W_{net} = \int_i^f F_{net} \, dx = m\int_i^f \dt{x}dv = m\int_i^f v \, dv \\
                            \tq W_{net} = \left[\frac{mv^2}{2}\right]_i^f = \Delta K
                        \end{equation}
                \end{itemize}
        \end{itemize}
    \subsection{Convervation of energy}
        \begin{itemize}
            \item Definition
                \begin{equation}
                    E_T=K+U=\textrm{constant}\\
                    \Delta E_T=\Delta K+\Delta U=0
                \end{equation}
                if there's no dissipative force
            \item Conservative forces requirements
                \begin{itemize}
                    \item the force has to depend only on the position of objects
                    \item work done by the force has to depend only on the initial and final states of a system (not on how it got from initial to final)
                \end{itemize}
            \item Work done by dissipative forces
                \begin{equation}
                    W_{diss}=\Delta E_T
                \end{equation}
                \begin{itemize}
                    \item Derivation
                        \begin{equation}
                            W_{net}=W_{cons}+W_{diss}=-\Delta U+W_{diss}=\Delta K\\
                            \tq W_{diss}=\Delta K+\Delta U=\Delta E_T
                        \end{equation}
                \end{itemize}
        \end{itemize}
    \subsection{Work done by conservative forces}
        \begin{enumerate}
            \item Spring force
                \begin{equation}
                    W=\frac{1}{2}kx_i^2-\frac{1}{2}kx_f^2=\frac{k}{2}\left(x_i^2-x_f^2\right)
                \end{equation}
            \item Weight
                \begin{equation}
                    W=-mg\Delta z
                \end{equation}
        \end{enumerate}
\section{Momentum}
\section{Angular momentum}
\section{Lagrangian method}